%------------------------------------------------------------------------------
%	PACKAGES AND OTHER DOCUMENT CONFIGURATIONS
%------------------------------------------------------------------------------
\documentclass[20pt]{article}
\usepackage[left=2.5cm, right=2.5cm, top=2.5cm, bottom=2.5cm]{geometry}
%\usepackage[T1]{fontenc}
\usepackage{xeCJK}
%\usepackage{fourier}
%\usepackage{times}
%\usepackage{lmodern}
\setCJKmainfont{SimSun}
\setCJKsansfont{SimHei}
\setCJKmonofont{SimHei}
\usepackage[english]{babel}
\usepackage{amsmath,amsfonts,amsthm}
\usepackage{graphicx}
\usepackage[usenames,dvipsnames]{color}
\usepackage{sectsty}
\usepackage{color}
\usepackage{indentfirst}
\usepackage{setspace} 
\usepackage{ulem}
%\allsectionsfont{ \centering \normalfont }
\allsectionsfont{ \normalfont }
\usepackage{fancyhdr}
\pagestyle{fancyplain}
\fancyhead{}
\fancyfoot[L]{}
\fancyfoot[C]{}
\fancyfoot[R]{\thepage}
\renewcommand{\headrulewidth}{0pt}
\renewcommand{\footrulewidth}{0pt}
\setlength{\headheight}{13.6pt}

\numberwithin{equation}{section}
\numberwithin{figure}{section}
\numberwithin{table}{section}
\setlength\parindent{0pt}
\setlength{\parindent}{2em}

%------------------------------------------------------------------------------
%	TITLE SECTION
%------------------------------------------------------------------------------
\newcommand{\horrule}[1]{\rule{\linewidth}{#1}}
\title{
        \normalfont \normalsize
        %\textsc{UNIVERSITY OF SCIENCE AND TECHNOLOGY OF CHINA} \\ [25pt]
        {UNIVERSITY OF SCIENCE AND TECHNOLOGY OF CHINA} \\ [25pt]
        \horrule{0.8pt} \\[0.4cm]
        \LARGE {ubuntu操作手册} \\
        %\Large {Part III : } %\Large {Review on ViscoElastic Media} \\
        \horrule{2pt} \\[0.5cm]
       }
\author{ZHOU Li}
\date  {\normalsize\today}
\begin{document}
\maketitle
\newpage
\tableofcontents
\newpage
%-----------------------------------------------------------------------------
\noindent\LARGE\section{\color{blue}如何杀死僵尸进程}
\normalsize{在UNIX 系统中,一个进程结束了,但是它的父进程没有等待(调用
wait / waitpid)它, 那么它将变成一个僵尸进程.  在fork()/execve()过程中,
假设子进程结束时父进程仍存在,而父进程fork()之前既没安装SIGCHLD信号处
理函数调用 waitpid()等待子进程结束,又没有显式忽略该信号,则子进程成为僵尸进程。}
%-----------------------------------------------------------------------------
\begin{spacing}{2.0} 
\noindent\large\color{Green}{Step 1.查看僵尸进程}
\end{spacing}
在终端输入
{\small\color{Brown}\begin{verbatim}
    ps -A -ostat,ppid,pid,cmd | grep -e '^[zZ]'
    或者
    ps -ef | grep defunct
\end{verbatim}}
\begin{spacing}{2.0}
\noindent\large\color{Green}{Step 2.杀死僵尸进程}
\end{spacing}
\normalsize{一般僵尸进程很难直接kill掉,不过您可以kill僵尸爸爸。父进程死后,僵尸
 进程成为”孤儿进程”,过继给1号进程init,init始终会负责清理僵尸进程.它产生的所
 有僵尸进程也跟着消失。}
{\color{Brown}{\begin{verbatim}
    kill -HUP `ps -A -ostat,ppid | grep -e ’^[Zz]‘ | awk ’{print $2}’`
    或者
    ps -ef | grep defunct | grep -v grep | awk '{print "kill -9 " $2,$3}'
\end{verbatim}}}
%-----------------------------------------------------------------------------
\noindent\LARGE\section{\color{blue}sed基本语法}
\normalsize{sed是一种非交互式的流编辑器,可动态编辑文件。所谓非交互式是说,sed和
传统的文本编辑器不同,并非和使用者直接互动,sed处理的对象是文件的数据流(称为
stream/流)。sed的工作模式是,比对每一数据行,若符合样式,就执行指定的操作。}
\noindent\large\subsection{\color{blue}sed替换}
{\color{Brown}{\begin{verbatim}
    sed -i 's/string1/string2/g' filename
\end{verbatim}}}
\noindent\large\subsection{\color{blue}sed删除}
{\color{Brown}{\begin{verbatim}
    1.删除某一段范围的数据行
    sed '1,4d' filename
    2.删除含某一字符或字符串的数据行
    sed '/string/d' filename
    3.删除空行
    sed '/^$/d' filename
\end{verbatim}}}

%-----------------------------------------------------------------------------
\noindent\LARGE\section{\color{blue}终端只显示当前目录}
\normalsize{在进行终端操作时,有时长长的路径名会把整行都占满,看起来和操作起来
    都很别扭,下面就讲一讲在.bashrc中怎么设置主提示符PS1为只显示当前目录}
{\color{Brown}{\begin{verbatim}
     vim ~/.bashrc
\end{verbatim}}}
\noindent{\color{blue}{找到}}
{\color{Brown}{\begin{verbatim}
    if [ "$color_prompt" = yes ]; then
      PS1='${debian_chroot:+($debian_chroot)}\[\033[01;32m\]\u@\h\[\033[00m\]:\
          [\033[01;34m\]\w\[\033[00m\]\$ '
    else
      PS1='${debian_chroot:+($debian_chroot)}\u@\h:\w\$ '
    fi
\end{verbatim}}}
将其中的w由小写改成大写,可以表示只显示当前目录名称。
对SUSE和Centos等,.bahsrc中是没有这些语句的,直接将以上修改后语句的添加进去即可。
\newline
\noindent{\color{blue}{最后执行}}
{\color{Brown}{\begin{verbatim}
    . .bashrc
\end{verbatim}}}
使修改后的环境变量生效。
\newline
\noindent{\color{blue}{PS:}}
\normalsize{如果想更加简洁,可以连@hostname都神略掉,直接变成usrname@currentdirectory,只需在.bashrc删掉@后面的$\backslash h$}
%-----------------------------------------------------------------------------
%-----------------------------------------------------------------------------
\noindent\LARGE\section{\color{blue}mpirun error}
\normalsize{组里新买了一台服务器,装的是suse,装好mpich3后,跑并行后出如下错误}
{\color{Brown}{\begin{verbatim}
     Fatal error in MPI_Init: Other MPI error, error stack:
         MPIR_Init_thread(498)..............:
         MPID_Init(177).....................: channel initialization failed
         MPIDI_CH3_Init(89).................:
         MPID_nem_init(320).................:
         MPID_nem_tcp_init(171).............:
         MPID_nem_tcp_get_business_card(418):
             MPID_nem_tcp_init(377).............: gethostbyname failed, simulator2 (errno 1)
\end{verbatim}}}
之前没用过suse,老是以为哪里设置不对,折腾了好久,之后发现是通讯设置的问题。解决方法如下:
{\color{Brown}{\begin{verbatim}
    sudo vim /etc/hosts
    修改如下:
    127.0.0.1        localhost.localdomain localhost
    ip               hostname
\end{verbatim}}}
第二行为对应的网络ip,hostname为机器名称。大部分linux发行版hosts文件已经配置好这些变量,
你也就省的改掉了。
%-----------------------------------------------------------------------------

\noindent\LARGE\section{\color{blue}git操作常见问题}
\normalsize{Git是一款免费、开源的分布式版本控制系统,用于敏捷高效地处理任何或小或大的项目}
\noindent\large\subsection{\color{blue}更换git托管}
\normalsize{更换git托管可能是一件主动也可能是一件被动的事情,比如我们组里的git远程服务器
    时不时的会重启那么一次,ip有时就会改变。搞得我就很被动。}
\begin{spacing}{2.0} 
\noindent\large\color{Green}{方法1:}
\end{spacing}
假设你的remote是origin,用git remote set\uline{ }url 更换地址,remote\uline{ }git\uline{ }address更换成你的新的仓库地址.
{\color{Brown}{\begin{verbatim}
    git remote set-url origin remote_git_address
\end{verbatim}}}
\begin{spacing}{2.0} 
\noindent\large\color{Green}{方法2:} 
\end{spacing}
执行如下命令:
{\color{Brown}{\begin{verbatim}
    git remote -v
    git remote rm origin
    git remote add origin git@222.195.xx.xx:name.git
    git fetch origin
\end{verbatim}}}
\noindent\large\subsection{\color{blue}撤销commit}
\normalsize{如果不满意当前的commit操作,可以对其进行撤销。
先使用git log查看git commit日志,找到需要回退的那次commit的哈希值,其实也就是指针了。
{\color{Brown}{\begin{verbatim}
    git reset --hard <commit_id>
    git push origin HEAD --force
\end{verbatim}}}
\noindent git reset –hard:彻底回退到某个版本,本地的源码也会变为上一个版本的内容;
\newline
git reset –soft:回退到某个版本,只回退了commit的信息,不会恢复到index file一级。
如果还要提交,直接commit即可;
\newline
git reset –mixed:此为默认方式,不带任何参数的git reset,即时这种方式,它回退到某个版本,
只保留源码,回退commit和index信息。
\end{document}

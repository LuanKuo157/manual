%------------------------------------------------------------------------------
%	PACKAGES AND OTHER DOCUMENT CONFIGURATIONS
%------------------------------------------------------------------------------
\documentclass[UTF8]{ctexart}
\usepackage[left=2.5cm, right=2.5cm, top=2.5cm, bottom=2.5cm]{geometry}
\usepackage[T1]{fontenc}
%\usepackage{fourier}
%\usepackage{times}
\usepackage{lmodern}
\usepackage[english]{babel}
\usepackage{amsmath,amsfonts,amsthm}
\usepackage{graphicx}
\usepackage[usenames,dvipsnames]{color}
\usepackage{sectsty}
\usepackage{color}
\usepackage{indentfirst} 
%\allsectionsfont{ \centering \normalfont }
\allsectionsfont{ \normalfont }
\usepackage{fancyhdr}
\pagestyle{fancyplain}
\fancyhead{}
\fancyfoot[L]{}
\fancyfoot[C]{}
\fancyfoot[R]{\thepage}
\renewcommand{\headrulewidth}{0pt}
\renewcommand{\footrulewidth}{0pt}
\setlength{\headheight}{13.6pt}

\numberwithin{equation}{section}
\numberwithin{figure}{section}
\numberwithin{table}{section}
\setlength\parindent{0pt}
\setlength{\parindent}{2em}

%------------------------------------------------------------------------------
%	TITLE SECTION
%------------------------------------------------------------------------------
\newcommand{\horrule}[1]{\rule{\linewidth}{#1}}
\title{
        \normalfont \normalsize
        %\textsc{UNIVERSITY OF SCIENCE AND TECHNOLOGY OF CHINA} \\ [25pt]
        {UNIVERSITY OF SCIENCE AND TECHNOLOGY OF CHINA} \\ [25pt]
        \horrule{0.8pt} \\[0.4cm]
        \LARGE {UBUNTU14.04 安装搜狗输入法和hp打印机驱动} \\
        %\Large {Part III : } %\Large {Review on ViscoElastic Media} \\
        \horrule{2pt} \\[0.5cm]
       }
\author{ZHOU Li}
\date  {\normalsize\today}
\begin{document}
\maketitle
\newpage
\tableofcontents
\newpage
%-----------------------------------------------------------------------------
\noindent\LARGE\section{\color{blue}这个还需要教程吗?}
\normalsize{的确在大多数情况下ubuntu14.04安装搜狗拼音输入法只需双击deb包,装hp打印机驱动只需运行
一下shell脚本就可,写个教程的确有点low.不过作为一个较新的发行版,对于一些三方软件14.04还
是显得有点娇贵,时不时会出些bug!}
\newline
\indent{前两天更新后,hp打印机驱动出了问题,显示的错误是找不到python gtk的库,按它的提示
装了几个库后,还是不行,我就想把python都给卸载了,重新配置python,就是在这时犯了一个剁手的
错误,我用sudo apt-get autoremove python* 来卸载,好吧现在问题来了,gnome, unity, desktop的
软件库依赖都被删了,很多应用都不行了,关机重启后连桌面系统都进不了,网络链接也出问题了,
没办法只能再命令行界面修复了.在修复好桌面环境后我发现之前的搜狗拼音也用不了,虽然ubuntu下的
搜狗拼音也有一些小bug,不过词库丰富,相比于其它输入法对我而言还是高效一些.下面我就介绍一下比较
完美的14.04下安装搜狗拼音和hp打印机驱动的方法.}
%-----------------------------------------------------------------------------
\noindent\LARGE\section{\color{blue}安装搜狗拼音}
\noindent\large{Step 1.去官网下载.deb包}
{\small\color{blue}\begin{verbatim}
    http://pinyin.sogou.com/linux/?r=pinyin
\end{verbatim}}

\noindent\large{Step 2.清理系统fcitx环境}
{\color{Brown}{\begin{verbatim}
    sudo apt-get purge fcitx*
    sudo purge autoremove    
\end{verbatim}}}

\noindent\large{Step 3.安装依赖库}
\newline\indent
\normalsize{相比于用软件中心来安装deb包,我更喜欢用安装器GDebi来安装deb包,和GDebi比起来,
 Ubuntu软件中心慢的太多了,CPU占用率也非常之高,而且应为不能自动解决依赖问题,总会出各种
 小bug。}
{\color{Brown}{\begin{verbatim}
    sudo apt-get install gdebi    
\end{verbatim}}}

\noindent\large{Step 4.用GDebi安装搜狗deb包}
{\color{Brown}{\begin{verbatim}
    sudo gdebi sogou_pinyin_linux_1.1.0.0037_amd64.deb
\end{verbatim}}}
\normalsize{然后重启,如果没有出现搜狗请在
系统设置->语言支持->键盘输入方式系统->选择fcitx}

\noindent\LARGE{\section{\color{blue}hp 打印机驱动安装}}
\normalsize ubuntu14.04默认的打印机驱动只适用部分hp打印机,如果不支持你的打印机型号,可以参考:
\normalsize{\color{blue}\begin{verbatim}
    http://hplipopensource.com/hplip-web/install/manual/distros/ubuntu.html
\end{verbatim}}
如果选则网络打印,需要配置好Apache HTTP服务器,安装appach2如下:
\normalsize{\color{blue}\begin{verbatim}
    sudo apt-get install apache2
\end{verbatim}}
%-----------------------------------------------------------------------------
\end{document}
%-----------------------------------------------------------------------------
%\section{Example of list (3*itemize)}
%\begin{itemize}
%	\item First item in a list
%		\begin{itemize}
%		\item First item in a list
%			\begin{itemize}
%			\item First item in a list
%			\item Second item in a list
%			\end{itemize}
%		\item Second item in a list
%		\end{itemize}
%	\item Second item in a list
%\end{itemize}
%-----------------------------------------------------------------------------
%\section{Example of list (enumerate)}
%\begin{enumerate}
%  \item First item in a list
%  \item Second item in a list
%  \item Third item in a list
%\end{enumerate}

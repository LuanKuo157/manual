%------------------------------------------------------------------------------
%	PACKAGES AND OTHER DOCUMENT CONFIGURATIONS
%------------------------------------------------------------------------------
\documentclass[UTF8]{ctexart}
\usepackage[T1]{fontenc}
%\usepackage{fourier}
%\usepackage{times}
\usepackage{lmodern}
\usepackage[english]{babel}
\usepackage{amsmath,amsfonts,amsthm}
\usepackage{graphicx}
\usepackage[usenames,dvipsnames]{color}
\usepackage{sectsty}
\usepackage{color}
\usepackage{indentfirst} 
%\allsectionsfont{ \centering \normalfont }
\allsectionsfont{ \normalfont }
\usepackage{fancyhdr}
\pagestyle{fancyplain}
\fancyhead{}
\fancyfoot[L]{}
\fancyfoot[C]{}
\fancyfoot[R]{\thepage}
\renewcommand{\headrulewidth}{0pt}
\renewcommand{\footrulewidth}{0pt}
\setlength{\headheight}{13.6pt}

\numberwithin{equation}{section}
\numberwithin{figure}{section}
\numberwithin{table}{section}
\setlength\parindent{0pt}
\setlength{\parindent}{2em}

%------------------------------------------------------------------------------
%	TITLE SECTION
%------------------------------------------------------------------------------
\newcommand{\horrule}[1]{\rule{\linewidth}{#1}}
\title{
        \normalfont \normalsize
        %\textsc{UNIVERSITY OF SCIENCE AND TECHNOLOGY OF CHINA} \\ [25pt]
        {UNIVERSITY OF SCIENCE AND TECHNOLOGY OF CHINA} \\ [25pt]
        \horrule{0.8pt} \\[0.4cm]
        \LARGE { The Best Configuration \LaTeX{} of Your UBUNTU PC } \\
        %\Large {Part III : } %\Large {Review on ViscoElastic Media} \\
        \horrule{2pt} \\[0.5cm]
       }
\author{ZHOU Li}
\date  {\normalsize\today}
\begin{document}
\maketitle
\newpage
\tableofcontents
\newpage
%-----------------------------------------------------------------------------
\LARGE{\color{blue}\section{Introduction}}
\normalsize{\LaTeX{} 在文档排版和公式编辑上的高效和强大功能就无需对理工科学生多言了!
对于linux下的\LaTeX{}初学者,如何配置一个最高效舒适的\LaTeX{}编译环境是一个普遍困扰的
问题。}
\newline
\indent{本文以UBUNTU12.04系统为例介绍如何配置一个好用的\LaTeX{}编译环境。UBUNTU其他版
本和CTENOS的安装一致或基本类似。}
\newline
\LARGE\section{\color{blue}为什么选用Texlive2014+Texmaker?}
\normalsize 推荐使用最新版本的Texlive,是因为它的软件包比较全,对中文的支持也比较好。为什么不用UBUNTU
自带的\LaTeX{}2009(或者sudo apt-get install这样安装的),用过的都知道,我就不说了。
\newline
\indent Texmaker是为了方便那些不习惯命令行编译的人提供的可视化\TeX{}文档编辑器。建议你自己到官
网下载最新版本的,如果你在UBUNTU自己的软件中心安装的话,你之前的安装可能就白费了,你的\LaTeX{}
又变成2009了。当然如果你习惯命令行编译,就不需要安装Texmaker。
\\
%-----------------------------------------------------------------------------
\LARGE{\section{清理系统环境}}
\noindent\Large{以删除texlive2013为例:}
{\color{Brown}{\begin{verbatim}
sudo apt-get purge texlive*
rm -rf /usr/local/texlive/2013 and rm -rf ~/.texlive2013
rm -rf /usr/local/share/texmf
rm -rf /var/lib/texmf
rm -rf /etc/texmf
sudo apt-get remove tex-common --purge
rm -rf ~/.texlive
\end{verbatim}}}


\noindent\LARGE{\section{下载TexLive及安装}}
\normalsize 科大学生推荐在自己的镜像源下载.iso镜像安装,下载速度奇佳。推荐下载
texlive2014-20140525.iso,因为下载的texlive2014.iso我自己的电脑安装过程中出现
问题,推测可能和硬件型号有关,未做过多测试,仅仅是推荐。下载地址:
{\small\color{blue}\begin{verbatim}
ftp://mirrors.ustc.edu.cn/CTAN/systems/texlive/Images/
\end{verbatim}}

\noindent\Large{挂载.iso及安装命令:}  
\normalsize{\color{blue}\begin{verbatim}
$ sudo mount -o loop texlive2014.iso /mnt 
$ cd  /mnt
$ sudo ./install-tl
\end{verbatim}}
安装过程中选i,就会按默认路径安装。即安装在$/\ usr/\ local/\ texlive$下。

\Large{配置环境变量:}  
\normalsize{\color{blue}\begin{verbatim}
$  vim ~/.bashrc 
添加:
export PATH=/usr/local/texlive/2014/bin/x86_64-linux:$PATH
export MANPATH=/usr/local/texlive/2014/texmf-dist/doc/man:$MANPATH
export INFOPATH=/usr/local/texlive/2014/texmf-dist/doc/info:$INFOPATH
执行
$ . ~/.bashrc
\end{verbatim}}
Texlive2014 安装完毕。
%-----------------------------------------------------------------------------
\end{document}
%-----------------------------------------------------------------------------
%\section{Example of list (3*itemize)}
%\begin{itemize}
%	\item First item in a list
%		\begin{itemize}
%		\item First item in a list
%			\begin{itemize}
%			\item First item in a list
%			\item Second item in a list
%			\end{itemize}
%		\item Second item in a list
%		\end{itemize}
%	\item Second item in a list
%\end{itemize}
%-----------------------------------------------------------------------------
%\section{Example of list (enumerate)}
%\begin{enumerate}
%  \item First item in a list
%  \item Second item in a list
%  \item Third item in a list
%\end{enumerate}

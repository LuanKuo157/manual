%------------------------------------------------------------------------------
%	PACKAGES AND OTHER DOCUMENT CONFIGURATIONS
%------------------------------------------------------------------------------
\documentclass[paper=a4, fontsize=11pt]{scrartcl}
\usepackage[T1]{fontenc}
%\usepackage{fourier}
%\usepackage{times}
\usepackage{lmodern}
\usepackage[english]{babel}
\usepackage{amsmath,amsfonts,amsthm}
\usepackage{graphicx}
\usepackage[usenames,dvipsnames]{color}
\usepackage{sectsty}
\usepackage{color}
%\allsectionsfont{ \centering \normalfont }
\allsectionsfont{ \normalfont }
\usepackage{fancyhdr}
\pagestyle{fancyplain}
\fancyhead{}
\fancyfoot[L]{}
\fancyfoot[C]{}
\fancyfoot[R]{\thepage}
\renewcommand{\headrulewidth}{0pt}
\renewcommand{\footrulewidth}{0pt}
\setlength{\headheight}{13.6pt}

\numberwithin{equation}{section}
\numberwithin{figure}{section}
\numberwithin{table}{section}
\setlength\parindent{0pt}

%------------------------------------------------------------------------------
%	TITLE SECTION
%------------------------------------------------------------------------------
\newcommand{\horrule}[1]{\rule{\linewidth}{#1}}
\title{
        \normalfont \normalsize
        %\textsc{UNIVERSITY OF SCIENCE AND TECHNOLOGY OF CHINA} \\ [25pt]
        {UNIVERSITY OF SCIENCE AND TECHNOLOGY OF CHINA} \\ [25pt]
        \horrule{0.8pt} \\[0.4cm]
        \LARGE {Instruction of Installation of Dependent software and Libraries to FD2DtopoEw} \\
        %\Large {Part III : } %\Large {Review on ViscoElastic Media} \\
        \horrule{2pt} \\[0.5cm]
       }
\author{ZHOU Li,Sun Yao-Chong, ZHANG Wei}
\date  {\normalsize\today}
\begin{document}
\maketitle
\newpage
\tableofcontents
\newpage
%-----------------------------------------------------------------------------
\LARGE{\section{Introduction}}
\normalsize{
This manual describes how to obtain and install NETCDF4 and MPICH3.NETCDF and MPICH has many options.Since we just
want to run the MPI and NETCDF program and applications, We will only offer a briefly 'standard'installation guide.
We also provide a brief guide to install matlab and intel compilers including icc and ifort in  this manual.  }
\\
%-----------------------------------------------------------------------------
\LARGE{\section{Install Intel Compilers }}
\Large{2.1}. You can download netcdf C and fortran package from 
{\color{Brown}\begin{verbatim}
http://software.intel.com/en-us/articles/intel-parallel-stu
dio-xe-2013-sp1-for-linux-update-2
\end{verbatim}}
\Large{2.2}. Unpack the tar file and enter to this directory,then run \textcolor{red}{$install.sh$} or \textcolor{red}{$install\_GUI.sh$}:
{\color{blue}\begin{verbatim}
$ tar -xvf parallel_studio_xe_2013_sp1_update2.tgz 
$ cd parallel_studio_xe_2013_sp1_update2
$ sudo sh intall.sh #or sudo sh install_GUI.sh
\end{verbatim}}
Then follow the prompts to complete installation.And you can refer detail installation information from the install
manumal Release\_Notes\_S  \\
tudio \_XE\_2013SP1\_L.pdf in the tar package.
\\
\Large{2.3}.Edit the .bashrc or /etc/profile to set the environment variables.
{\color{Brown}\begin{verbatim}
$ vim  ~/.profile or sudo vim /etc/profile
\end{verbatim}}
Then add the commands below:
{\color{blue}\begin{verbatim}
if [ -f /opt/intel/bin/ifortvars.sh ]; then
     source /opt/intel/bin/ifortvars.sh intel64
fi
if [ -f  /opt/intel/bin/iccvars.sh ]; then
    source /opt/intel/bin/iccvars.sh intel64
fi
\end{verbatim}}
Then let the configured environment variables become effective:
{\color{blue}\begin{verbatim}
$ source ~/.profile  or  . /etc/profile
\end{verbatim}}
%-----------------------------------------------------------------------------------------------------------------------------------------

\LARGE{\section{Install Matlab and Snctools}}
\Large{3.1}. Dowload the installation *.iso file .
\\
\Large{3.2}. Mount the *.iso file :
{\color{blue}\begin{verbatim}
$ sudo mount -o loop Mathworks.Matlab.R2012a.UNIX.iso /mnt 
$ cd  /mnt
$ sudo ./install
\end{verbatim}}
Then follow the prompts to complete installation.  
You may got the following error when you try to launch MATLAB in Ubuntu 12.04 64bit: 
{\color{blue}\begin{verbatim}
/opt/MATLAB/R2011a/bin/util/oscheck.sh: 605: /lib/libc.
so.6: not found
\end{verbatim}}
To resolve this problem, open a Terminal window and use the commands below:
{\color{blue}\begin{verbatim}
$ sudo ln -s /lib/x86_64-linux-gnu/libc-2.15.so /lib64/ 
libc.so.6
\end{verbatim}}
This command tries to create a link to the newest libc(2.15) dynamic library located under
/lib/x86\_64-linux-gnu/ and put it to /lib64/ or /lib/ named as libc.so.6, 
which is required by Matlab. 
\\
\Large{3.3}.Edit the ~/.profile  or /etc/profile to set the environment variables.
{\color{Brown}\begin{verbatim}
$ vim  ~/.profile  or sudo vim /etc/profile
\end{verbatim}}
Then add the commands below into ~/.profile  or /etc/profile:
{\color{blue}\begin{verbatim}
PATH="$PATH:/opt/Matlab/R2012a/bin"
\end{verbatim}}
Then let the configured environment variables become effective:
{\color{blue}\begin{verbatim}
$ source ~/.profile  or . /etc/profile
\end{verbatim}}
\Large{3.4}.Install Snctools toolbox Plug-in
\\
\large{Setep 1}. Download mexcdf Plug-in from
{\color{Brown}\begin{verbatim}
 http://mexcdf.sourceforge.net/
\end{verbatim}}
\large{Setep 2}. Unpack the zip file and copy it to matlab toolbox directory:
{\color{Brown}\begin{verbatim}
$ unzip mexcdf.r4053.zip
$ cp mexcdf /opt/Matlab/R2012a/toolbox
\end{verbatim}}
\large{Setep 3}.Add the mexcdf directory into your matlab path.
{\color{Brown}\begin{verbatim}
$ Menu Files->Set Path then->Add with Subfolders
or 
$suo vim /opt/Matlab/R2012a/toolbox/local/pathdef.m
\end{verbatim}}
%-------------------------------------------------------------------------------------------------------------------------------
\LARGE{\section{Install NETCDF4}}

\Large{4.1}. You can download netcdf C and fortran package from 
{\color{Brown}\begin{verbatim}
http://www.unidata.ucar.edu/downloads/netcdf/index.jsp
\end{verbatim}}

\Large{4.2}. Install NetCDF C library
\\
\large{4.2.1}. Unpack the tar file and enter to this directory:
\\
{\color{blue}\begin{verbatim}
$ tar -xvf netcdf-4.3.1.1.tar.gz
$ cd netcdf-4.3.1.1
\end{verbatim}}
\large{4.2.2}.For gcc version:
\\
{\color{blue}\begin{verbatim}
$ export CC=gcc
$ sudo ./configure --prefix=/opt/netcdf4/4.3.1-gcc --disable-dap 
--disable-netcdf-4
$ make check
$ make install
\end{verbatim}}
The explanation of the above commands:
{\color{green}\begin{verbatim}
$ ./configure       
 \end{verbatim}}
The configuration script will set up the Makefiles that will be used to build the NetCDF libraries and utilities. 
It will also set up the installation directory for the default location of /usr/local.
If you would like to install the libraries in another location, use this configure command:
{\color{green}\begin{verbatim}
$ ./configure --prefix=/your/desired/install/directory)
\end{verbatim}}
And if you want to create a directory in the root directory , you need add 'sudo' in front of ./configure .
{\color{green}\begin{verbatim}
$ make----Make the libraries
\end{verbatim}}
When the configuration step completes successfully, you can build the libraries using `make'.
{\color{green}\begin{verbatim}
$ make check-----Testing NetCDF on Linux*
\end{verbatim}}
You can test your NetCDF libraries using 'make check'.
{\color{green}\begin{verbatim}
$ make install------Installing NetCDF on Linux*
\end{verbatim}}
Install NetCDF libraries using 'make install'.
 This will install the NetCDF libraries, include files, and utilities in the default location of /usr/local or the location specified in the configuration 
 step with the --prefix= option.
 \\
 And you can also run  the shell script \textcolor{red}{$netcdfc\_install\_allin1.sh$} from the attachment to install NETCDF4.
 \\
\large{4.2.3}.For icc version:
\\
 {\color{blue}\begin{verbatim}
  $   export CC=icc
  $   export LDFLAGS=-static-intel
  $   sudo ./configure --prefix=/opt/netcdf4/4.3.1-icc --disable-dap
  --disable-netcdf-4
  $   make check
  $   make install
 \end{verbatim}}
 Any detail information please see part 2.2.And you can also run  the shell script 
 \textcolor{red}{$netcdfc\_ \\
 install\_allin1.sh$} from the attachment to install NETCDF4 for icc compiler.
 \newline
 \Large{4.3}. Install NetCDF Fortran library
 \newline
\large{4.3.1}. Unpack the tar file and enter to this directory:
{\color{blue}\begin{verbatim}
$ tar -xvf netcdf-fortran-4.2.tar.gz
$ cd netcdf-fortran-4.2
\end{verbatim}}
\large{4.3.2}.For ifort version:
\\
{\color{blue}\begin{verbatim}
$   export LD_LIBRARY_PATH=/opt/netcdf4/4.3.1-icc/lib:${LD_LIBRARY_PATH}
$   export CC=icc
$   export FC=ifort
$   export CPPFLAGS=-I/opt/netcdf4/4.3.1-icc/include
$   export LDFLAGS=-L/opt/netcdf4/4.3.1-icc/lib
$   sudo ./configure --prefix=/opt/netcdf4/4.2.2-ifort
$   make check
$   sudo make install
\end{verbatim}}
If you want to install netcdf on other fortran compiler,please see and run  the shell script 
 \textcolor{red}{$netcdff\_install\_allin1.sh$} from the attachment.
%-------------------------------------------------------------------------------------------

 \LARGE{\section{Install MPICH3}}
\Large{5.1}. You can download MPICH from 
{\color{Brown}\begin{verbatim}
http://www.mpich.org/
\end{verbatim}}
\large{5.2}. Unpack the tar file and enter to this directory:
{\color{blue}\begin{verbatim}
$ tar -xvf mpich-3.1.tar.gz
$ cd mpich-3.1
\end{verbatim}}
{\color{blue}\begin{verbatim}
 $ make clean
 $ make V=1
 $ sudo ./configure \
  CC=/usr/bin/gcc\
  CXX=/usr/bin/gcc\
  F77=/opt/intel/composer_xe_2013_sp1.2.144/bin/intel64/ifort \
  FC=/opt/intel/composer_xe_2013_sp1.2.144/bin/intel64/ifort \
  --enable-fast=all MPICHLIB_CFLAGS=-O3 \
  MPICHLIB_FFLAGS=-O3 \
  MPICHLIB_CXXFLAGS=-O3 \
  MPICHLIB_FCFLAGS=-O3 \
  --enable-debuginfo \
  --enable-shared \
  -prefix=/opt/mpich/3.1-intel
 $ make
 $ make check
 $ sudo  make install
 \end{verbatim}}
 You can also install MPICH3 by running  the shell script \textcolor{red}{$mpich3\_install.sh$}.Since the install.pdf of MPICH writes very  clear ,
 you can refer from it,so I don't want to describe more
 \LARGE{\section{Contact Us}}
\normalsize{This is just a draft. If you have any questions or advice, please contact us.}
%\begin{figure}
% \centering
% \includegraphics[width=60mm,height=60mm]{distributed.png}
%\end{figure}
%\newpage
%-----------------------------------------------------------------------------
\end{document}
%-----------------------------------------------------------------------------
%\section{Example of list (3*itemize)}
%\begin{itemize}
%	\item First item in a list
%		\begin{itemize}
%		\item First item in a list
%			\begin{itemize}
%			\item First item in a list
%			\item Second item in a list
%			\end{itemize}
%		\item Second item in a list
%		\end{itemize}
%	\item Second item in a list
%\end{itemize}
%-----------------------------------------------------------------------------
%\section{Example of list (enumerate)}
%\begin{enumerate}
%  \item First item in a list
%  \item Second item in a list
%  \item Third item in a list
%\end{enumerate}
